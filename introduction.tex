% !TeX root = heatdiscrete.tex
\section{Introduction}
\label{sec:introduction}

Suppose that some initial heat configuration 
$u\colon\Omega\to\realspos$ is given over a finite 
space $\Omega$ and the configuration evolves 
according to the map $w\mapsto Sw$ in each time step 
$t=0,1,\ldots$, for some symmetric stochastic matrix 
$S\colon \Omega\times\Omega\to\realspos\,$.
Assume that we are interested in the amount of heat 
contained in a certain region $R\subseteq \Omega$ 
and how this quantity changes over time. In notation, 
assuming $\lnorm{u}{2}=1$ for normalization purposes 
and $v(x) \defeq 1/|R|^{1/2}$ for $x\in R$ 
(and 0 otherwise), we would like to understand how
\begin{align*}
m_t \defeq \iprod{v}{S^tu}
\end{align*}
changes as a function of $t$. In this paper we 
derive local bounds that $\set{m_t}_{t=0}^\infty$ 
must obey for any $S,u$ and $v$ satisfying the 
symmetry, magnitude and positivity constraints above 
(in fact our bounds work for any countable
$\Omega$, arbitrary non-negative unit vector $v$ 
and symmetric non-negative $S$).
Our first bound $m_{t+2}\ge \smash{m_t^{1+2/t}}$ 
answers a question of Blakley and Dixon from 1966
(see \autoref{conj:blakley-dixon} below) which was later 
conjectured independently by 
Erdős and Simonovits in 1982 also 
(see \autoref{conj:erdos-simonovits} in 
\autoref{sec:relatedwork}).

Moreover we establish a tight connection between such 
bounds and the well-studied $k$-Hamming distance problem
\cite{PangG1986, Yao2003, CormodePS2000, BarYossefJKK2004,
GavinskyKW2004, HuangSZZ2006,
BlaisBM2012, BuhrmanGMW2012, BlaisBG2014, AmbainisGSU2015} 
and the $k$-Hamming weight problem
\cite{AdaFH2012, BlaisK2012, BuhrmanGMW2012}
and obtain the first tight bounds for respectively
the communication complexity and parity decision tree 
complexity of them.
Our tight $\Omega(k \log (k/\delta))$ lower bound 
for the $\delta$-error communication complexity of 
the $k$-Hamming distance problem (that applies 
whenever $k^2< \delta n$) answers affirmatively a 
conjecture stated in \cite{BlaisBG2014} 
(Conjecture 1.4).
Prior to our work, the best impossibility results 
for this problem were an $\Omega(k\log^{(r)}k)$ bits 
lower bound ($\smash{\log^{(r)}z}$ being the $r$ nested 
applications of logarithm) that applies to any randomized $r$-round 
communication protocol \cite{SaglamT2013}, 
and an $\Omega(k\log (1/\delta))$ lower bound that 
applies to any $\delta$-error randomized protocol for 
$k <\delta n$ \cite{BlaisBG2014}.

Our parity decision tree lower bound shows that any 
$\delta$-error parity decision tree solving the 
$k$-Hamming weight problem has size 
$\exp\Omega\nparen{k\log (k/\delta)}$, 
which directly implies an $\Omega(k\log (k/\delta))$ 
bound on the depth of any such decision tree. 
Previously no nontrivial lower bound was known for 
the parity decision tree size of this problem and 
an $\Omega(k\log (1/\delta))$ bound on the parity 
decision tree depth followed from the communication 
complexity bound of \cite{BlaisBG2014}. Prior to 
\cite{BlaisBG2014}, the best bound on the parity 
decision tree depth was $\Omega(k)$, derived in 
\cite{BlaisBM2012} and \cite{BlaisK2012}.

Either by combining our communication complexity lower 
bound with the reduction technique developed in 
\cite{BlaisBM2012} or by combining our parity decision 
tree lower bound with a reduction given in 
\cite{BhrushundiCK2014}, one obtains an 
$\Omega(k\log (k/\delta))$ bound for any (potentially adaptive) 
property tester for the $\delta$-error probability 
$k$-linearity testing problem. This establishes the 
correct bound for this problem which was studied extensively 
\cite{FischerLNRRS2002, Goldreich2010, BlaisK2012, 
BhrushundiCK2014,BuhrmanGMW2012, BlaisBM2012}
since \cite{FischerLNRRS2002} or earlier.


\subsection{Motivating our bounds on $m_t$}
We would like to provide some intuition as to why one should expect
\begin{align}
m_{t+2}    &\ge m_t^{1+2/t}, \text{ and}\label{eq:bd66}\\
m_{t+2}    &\ge m_t^{1+2/t}\cdot \min\set{t^{1-\eps}, 
            \delta\frac{m_t^{1-2/t}}{m_{t-2}}}
\end{align}
to hold for appropriate $\eps,\delta$.
Recalling that $S$ is a symmetric matrix with maximum 
eigenvalue 1, we may write $S=QDQ^\transpose$
for an orthonormal matrix $Q$ having columns $q_x$, $x\in\Omega$
and a diagonal matrix $D$ with entries $\lambda_x \le 1$, 
$x\in \Omega$. Plugging this into 
$m_t=\iprod{v}{S^tu}$, we get
\begin{align}
\label{eq:spectralsum}
m_t = \sum_{x\in\Omega}\lambda_x^t\iprod{u}{q_x}\iprod{v}{q_x}.
\end{align}
For sake of analogy let us drop our assumption that $S, u, v$
are coordinate-wise nonnegative for a moment but instead assume
that each summand in the right hand side of
\autoref{eq:spectralsum}
is nonnegative by some coincidence. In this case we can
consider $\set{m_t}_{t=0}^\infty$ as the moment sequence of a 
random variable supported on $[0,1]$ that takes the
value $|\lambda_x|$ with probability 
$|\iprod{u}{q_x}\iprod{v}{q_x}|$ and the value 0 with
probability
$1-\sum_x|\iprod{u}{q_x}\iprod{v}{q_x}|$ (which is nonnegative
by Cauchy-Schwarz inequality). 
This would imply that $\set{m_t}_{t=0}^\infty$ is 
{\em completely monotone} by Hausdorff's characterization 
\cite{Hausdorff1921} and therefore log-convex 
(e.g., \cite{NiculescuP2005}, Section 2.1, Example 6).

One particular implication of the log-convexity of 
$\set{m_t}_{t=0}^\infty$, that 
$\frac{1}{t}\log m_t + \frac{t-1}{t}\log m_0\ge \log m_1$,
when combined with the fact $0\le m_0\le 1$ (that follows from our
assumption on the terms of \autoref{eq:spectralsum}), leads to
$m_t\ge m_1^t$.
In 1958, Mandel and Hughes showed that if $u=v$, rather
surprisingly, one can trade the assumption that the summands of
\autoref{eq:spectralsum} are nonnegative with the assumption
that $S$ and $u=v$ are coordinate-wise nonnegative and still
obtain the conclusion $m_t\ge m_1^t$:
\begin{theorem}[Mandel and Hughes \cite{MandelH1958}]
\label{thm:blakley-roy}
Let $u$ be a nonnegative unit vector and 
$S$ be a symmetric matrix with nonnegative entries. For an integer%
\footnote{Since $u=v$ here, the summands inside 
\autoref{eq:spectralsum} are nonnegative when $t$ is even  
so this theorem is most interesting for $t$ odd.}
$t\ge1$ we have $\iprod{u}{S^tu}\ge \iprod{u}{Su}^t$.
\end{theorem}


A more general implication of the log-convexity of 
$\set{m_t}_{t=0}^\infty$ and that $m_0\le 1$ is that for 
$k\ge t$, $\frac{t}{k}\log m_k + \frac{k-t}{k}\log m_0\ge \log m_t$,
therefore $m_k^t\ge m_t^k$. In 1966,
Blakley and Dixon \cite{BlakleyD1966} investigated 
whether $m_k^t\ge m_t^k$ holds in the case $u=v$ when the nonnegativity
assumption on the summands of \autoref{eq:spectralsum} is
replaced by the coordinate-wise nonnegativity of $S$, $u=v$.
They note that the inequality $m_k^t\ge m_t^k$ 
fails when $k$ and $t$ have different 
parity and otherwise holds true under the restriction 
$m_t\ge e^{-4t}$.
While the following is not explicitly stated as a conjecture 
in \cite{BlakleyD1966}, they write
\begin{quote}
if $t> 1$, [...] we cannot show that the inequality
\autoref{eq:bd66} holds for each nonnegative $|\Omega|$-vector
$u$ if $S$ is nonnegative.
\end{quote}
so with the earlier caveat we attribute the following to Blakley and
Dixon \cite{BlakleyD1966}:

\begin{conjecture}[Blakley and Dixon \cite{BlakleyD1966}]
\label{conj:blakley-dixon}
Let $S\colon\Omega\times\Omega\to\realspos$ be a symmetric matrix
with nonnegative entries and let $u\colon\Omega\to\realspos$ be a
nonnegative unit vector. For positive integers 
$k\ge t$ of the same parity, we have
  \begin{align*}
    \iprod{u}{S^ku}^t\ge \iprod{u}{S^tu}^k.
  \end{align*}
\end{conjecture}

In \autoref{sec:monotonicity} we prove the following theorem which
shows that a generalization of \autoref{conj:blakley-dixon}
holds true.
\begin{theorem}
\label{thm:blakley-dixon}
Let  $S\colon\Omega\times\Omega\to\realspos$ be a symmetric matrix
with nonnegative entries and 
$u,v\colon\Omega\to\realspos$ be
nonnegative unit vectors. For positive integers 
$k\ge t$ of the same
parity, we have
  \begin{align*}
    \iprod{v}{S^ku}^t\ge \iprod{v}{S^tu}^k.
  \end{align*}
\end{theorem}
It goes without saying that \autoref{eq:bd66} is 
equivalent to \autoref{thm:blakley-dixon} as we can rearrange 
\autoref{eq:bd66} to $\smash{m_{t+2}^{1/(t+2)}\ge m_t^{1/t}}$ and apply
it iteratively to obtain 
$\smash{m_k^{1/k}\ge\cdots\ge m_{t+2}^{1/(t+2)}\ge m_t^{1/t}}$ whenever 
$k\ge t$ and $k,t$ have the same parity. Moreover, while defining 
\autoref{eq:bd66} we assumed $S$ to be substochastic only to illustrate our
interpretation of the inequality: indeed any nonnegative $S$ can be scaled 
to be substochastic as both sides of \autoref{eq:bd66} are 
$(t+2)$-homogeneous in $S$.

In \autoref{thm:blakley-roy} and \autoref{thm:blakley-dixon} 
we observed that increasingly more general implications 
of the log-convexity of $\set{m_t}_{t=0}^\infty$ can be 
derived by only assuming the coordinate-wise nonnegativity 
of $S,u$ and $v$. One may naturally wonder if the 
coordinate-wise nonnegativity of $S,u$ and $v$ implies
the log-convexity of $\set{m_t}_{t=0}^\infty$ in its entirety. 
Unfortunately the following example shows that this is 
far from the truth.
\begin{figure}[H]
\centering
\begin {tikzpicture}[]
\tikzstyle{state}=[circle, draw, inner sep=0pt, minimum size=27pt]
\node[state] (A){$0$};
\node[state] (B) [right =of A] {$1$};
\node[state] (C) [right =of B] {$2$};
\node        (E) [right =of C] {$\cdots$};
\node[state] (J) [right =of E] {$t-1$};
\node[state] (K) [right =of J] {$t$};
\path (A) edge node[below] {$\eps$} (B);
\path (B) edge node[below] {$\eps$} (C);
\path (C) edge node[below] {$\eps$} (E);
\path (E) edge node[below] {$\eps$} (J);
\path (J) edge node[below] {$\eps$} (K);
\end{tikzpicture}
\caption{$\Omega=\set{0,1,\ldots, t}$, $S(i,i+1)=S(i+1,i)=\eps$ for $i=0,\ldots,t-1$.}
\label{fig:ex1}
\end{figure}
Consider the transition matrix $S$ on 
$\Omega=\set{0,1,\ldots,t}$ such that 
$S(i,i+1)=S(i+1,i)=\eps$ for 
$i=0,1,\ldots,t-1$ and $S(i,j)=0$ elsewhere.
Let $u$ and $v$ be the point masses respectively on 
states $0$ and $t$; namely 
$u=[1,0,\ldots,0]^\transpose$ and 
$v=[0,0,\ldots,1]^\transpose$.
We have $m_{t-2}=0$, $m_t=\eps^t$ and $m_{t+2}=t\eps^{t+2}$. 
Therefore $m_{t-2}m_{t+2} = 0 \not\ge \eps^{2t} = m_t^2$.
In this example the log-convexity breaks 
(in the strongest possible way) because the states $0$ 
and $t$ are separated by $t$ hops according to $S$ and 
the point mass at state $0$ cannot reach state $t$ 
before the $t$th time step. 

Our next theorem shows that such reachability issues 
are essentially the only way the log-convexity property 
can fail to hold:
\begin{theorem}
\label{thm:main}
For every $\eps>0$ there is a $\delta>0$ such that 
for any symmetric matrix 
$S\colon\Omega\times\Omega\to\realspos$ and unit 
vectors $u,v\colon\Omega\to\realspos$ with nonnegative 
entires, defining $m_t$ as before, we have
\begin{align}
\frac{m_{t+2}}{m_{t}^{1+2/t}}\ge
\min\set{t^{1-\eps}, \delta\frac{m_t^{1-2/t}}
{m_{t-2}}},\quad\quad\forall t\ge 2.
\label{eq:insidemain}
\end{align}
\end{theorem}
In other words, \autoref{thm:main} shows that one can recover a
truncated version of the log-convexity of
$\set{m_t}_{t=0}^\infty$ from just the coordinate-wise
nonnegativity assumption of $S,u$ and $v$.
We stress that \autoref{thm:main} is tight 
up to the appearance of $\eps$ and the choice of 
$\delta=\delta(\eps)$. A direct calculation
on \autoref{fig:ex1} for time steps $t, t+2, t+4$ 
shows that \autoref{eq:insidemain} cannot be improved to
\begin{align*}
\frac{m_{t+2}}{m_{t}^{1+2/t}}\ge
\min\set{t^{1-2/t}, 
\nparen{\frac{1+\eta}{2}}\frac{m_t^{1-2/t}}{m_{t-2}}}
\end{align*} for $\eta>0$.

\subsection{Related work on $m_t$}
\label{sec:relatedwork}
Almost simultaneously with the work of Mandel and Hughes
\cite{MandelH1958}, Mulholland and Smith also prove
\autoref{thm:blakley-roy} in \cite{MulhollandS1959} and 
moreover they characterize the equality conditions of 
the inequality. Independently, in 1965, Blakley and Roy 
\cite{BlakleyR1965} prove the same inequality and
characterize the equality conditions and 
\cite{London1966} provides an alternative proof to that 
of \cite{MulhollandS1959} in 1966.
We remark that \autoref{thm:blakley-roy} is 
most commonly referred to as the Blakley-Roy bound
or ``Sidorenko's conjecture for walks''.
Note these results show that \autoref{conj:blakley-dixon} 
is true whenever $t$ divides $k$. 
Finally in 2012, Pate shows that $m_t\ge m_1^t$ 
without the restriction $u=v$:

\begin{theorem}[Pate \cite{Pate2012}]
Let $S\colon\Omega\times\Omega\to\realspos$ be a 
symmetric matrix
with nonnegative entries and let 
$u,v\colon\Omega\to\realspos$ be
nonnegative unit vectors.
It holds that
\begin{align*}
\iprod{v}{S^{2t+1}u} \ge \iprod{v}{Su}^{2t+1},
\end{align*}
with equality if and only if 
$\iprod{v}{S^{2t+1}u} =0$ or 
$Su=\lambda v$ and 
$Sv=\lambda u$ 
for some $\lambda\in\realspos$.
\end{theorem}
This result already shows that 
\autoref{thm:blakley-dixon} is true when $t$ divides $k$ 
but such a bound does not have any implications for 
our applications in complexity theory. In \cite{ErdosS1982},
Erdős and Simonovits conjecture the following.
\begin{conjecture}
[Erdős and Simonovits \cite{ErdosS1982}, Conjecture 6]
\label{conj:erdos-simonovits}
For a graph $G=(V,E)$, let $w_k(G)$ be the number length 
$k$ walks in $G$ divided by $|V|$.
For an undirected graph $G$, we have 
$w_k(G)^t\ge w_t(G)^k$ for $k>t$ of the same parity.
\end{conjecture}
This conjecture was recalled in a recet book
\cite{Taubig2017} by Täubig as Conjecture~4.1.
Note that \autoref{conj:erdos-simonovits} is a 
specialization of \autoref{conj:blakley-dixon} to
$S$ having 0-1 entries and  
$u=\mathbf{1}/\sqrt{|V|}$ therefore our 
\autoref{thm:blakley-dixon} verifies 
\autoref{conj:erdos-simonovits} as well.


\subsection{Our results in complexity theory}
\label{sec:results-cc}
Here we list our results in complexity theory; see
\autoref{sec:applications} for the definition
of the models and the problems.
The following theorem 
(which was already known \cite{BlaisBG2014})
is a consequence of \autoref{thm:blakley-dixon} 
and uses the standard corruption technique in 
communication complexity.
\begin{theorem}
\label{thm:klogdelta}
Any two party $\delta$-error randomized protocol
solving the $k$-Hamming distance problem 
over length-$n$ strings communicates 
at least $\Omega(k\log (1/\delta))$ 
bits for $k^2\le \delta n$.
\end{theorem}
The next is our main result for the communication 
complexity of the $k$-Hamming distance problem
and is a consequence of \autoref{thm:main}. 
This result cannot be obtained by the standard 
corruption technique and
requires a suitable modification similar to 
\cite{Sherstov2012}.
\begin{theorem}
\label{thm:klogk}
Any two party $\delta$-error randomized protocol
solving the $k$-Hamming distance problem 
over length-$n$ strings communicates 
at least $\Omega(k\log (k/\delta))$ 
bits for $k^2\le \delta n$.
\end{theorem}
\begin{theorem}
\label{thm:paritysize}
Any $\delta$-error parity decision tree 
deciding the $k$-Hamming weight predicate 
over length-$n$ strings
has size $\exp \Omega\nparen{k\log (k/\delta)}$
for $k^2< \delta n$.
\end{theorem}
\begin{corollary}
\label{cor:propertytest}
Any $\delta$-error probability 
property tester for 
$k$-linearity requires 
$\Omega(k\log (k/\delta))$ queries.
\end{corollary}

Note the bound $m_k^t\ge m_t^k$ obtained in 
\cite{BlakleyD1966} under the condition $m_t\ge e^{-4t}$ 
does not have any implications for the communication 
complexity of the $k$-Hamming distance problem as our 
reduction crucially uses the fact that $u$ and $v$ are 
arbitrary, however it does lead to an $\exp\Omega(k)$ 
lower bound for the parity decision tree size of the 
$k$-Hamming weight problem when combined with our 
reduction.

\begin{remark}
Note that in \autoref{thm:blakley-dixon}, when $u=v$ and either
both $k,t$ are even or $S$ is positive semidefinite, the
summands of \autoref{eq:spectralsum} become nonnegative and the
inequality holds trivially. For our application in communication
complexity we crucially use the fact that $u$ and $v$ are
arbitrary and for our application in parity decision trees, one
can do away with $u=v$ but only at the expense of having to
choose $k,t$ odd. 
In both results the $S$ we choose has eigenvalues $1$ and $-1$ 
with equal multiplicities and therefore far away from being positive 
semidefinite. In either case, the implications of
\autoref{thm:blakley-dixon} in complexity theory follow 
from the interesting cases of this theorem.
\end{remark}
